% lualatex quizogbyt.tex [database.csv]
% Hvor csv filen skal have to kolonner [Ord] og [Forklaring]
\documentclass[a4paper,oneside]{memoir}
\usepackage{luacode}
\directlua{require("luafunctions.lua")}
\begin{luacode}
	-- Reading commandline arguments
    db_path = arg[2] or "sample.csv"
	number = arg[3] or 2
	colorsPath = "colors.csv"
\end{luacode}

\usepackage{ifdraft}

\settypeblocksize{276mm}{200mm}{*} % Højde*bredde på side
\setlrmargins{*}{*}{1}
\setulmargins{*}{*}{0.5}
\setheaderspaces{*}{3mm}{*}
\setheadfoot{\baselineskip}{0pt}
\checkandfixthelayout
\pagestyle{empty}
\makeatletter
\makeoddhead{empty}{\@title, sæt \arabic{CardSet}}{\today}{\thepage}
\makeevenhead{empty}{\thepage}{\today}{\@title, sæt \arabic{CardSet}}
\newcommand{\Titel}{\@title}
\setlength{\parindent}{0mm}
\makeatother
\ifdraft{\usepackage{showframe}}{}

\begin{luacode}
	db = readCSV( db_path )
\end{luacode}

\usepackage{graphicx}
\graphicspath{{./Billeder/}}
\usepackage{pgffor}
\usepackage{tcolorbox}
\tcbuselibrary{skins,hooks}

\newcounter{CardSet}
\newcounter{CardNumber}[CardSet]

\title{\directlua{tex.print(db_path)}}

\begin{document}
\tcbset{enhanced,width=(\linewidth)/3,nobeforeafter,arc=4mm,colback=white,
left=1mm,right=1mm,top=6mm,bottom=6mm,middle=6mm,equal height group=kort,
valign upper=center, valign lower=bottom, split=1,
enlarge top by=1mm, enlarge bottom by=1mm,grow to left by=-1mm, grow to right by=-1mm}%
\ifdraft{\tcbset{show bounding box}}{}%
\directlua{tex.print(loadColors(colorsPath,number))}%
\foreach \farve in \farver {% Loop der laver flere sæt kort, med forskellig farve
\stepcounter{CardSet}%
\tcbset{colframe=\farve}%
\loop \stepcounter{CardNumber} \ifnum \value{CardNumber} < 5%
\begin{tcolorbox}
	\directlua{ tex.print( genCardcontent( tex.count['c@CardNumber'] ) ) }
\end{tcolorbox}%
\allowbreak%
\repeat%
\cleardoublepage
}% End color series
\end{document}



