\documentclass[a4paper,oneside]{memoir}
\usepackage[utf8]{inputenc}

\settypeblocksize{276mm}{200mm}{*} % Højde*bredde på side
\setlrmargins{*}{*}{1}
\setulmargins{*}{*}{0.5}
\setheaderspaces{*}{3mm}{*}
\checkandfixthelayout
\pagestyle{plain}
\makeatletter
\makeoddhead{plain}{\@title, sæt \arabic{KortSet}}{\today}{\thepage}
\makeevenhead{plain}{\thepage}{\today}{\@title, sæt \arabic{KortSet}}
\newcommand{\Titel}{\@title}
\setlength{\parindent}{0em}
\makeatother

\usepackage{datatool}
\DTLsetseparator{,}
\DTLloaddb{kort}{dyr.csv}

\usepackage{graphicx}
\graphicspath{{./Billeder/}}
\usepackage{pgffor}
\usepackage{tcolorbox}
%Highlight
\definecolor{AluminiumHighlight}{HTML}{eeeeec}
\definecolor{ButterHighlight}{HTML}{fce94f}
\definecolor{ChameleonHighlight}{HTML}{8ae234}
\definecolor{OrangeHighlight}{HTML}{fcaf3e}
\definecolor{ChocolateHighlight}{HTML}{e9b96e}
\definecolor{SkyBlueHighlight}{HTML}{729fcf}
\definecolor{PlumHighlight}{HTML}{ad7fa8}
\definecolor{SlateHighlight}{HTML}{888a85}
\definecolor{ScarletRedHighlight}{HTML}{ef2929}
%Normal
\definecolor{Aluminium}{HTML}{d3d7cf}
\definecolor{Butter}{HTML}{edd400}
\definecolor{Chameleon}{HTML}{73d216}
\definecolor{Orange}{HTML}{f57900}
\definecolor{Chocolate}{HTML}{c17d11}
\definecolor{SkyBlue}{HTML}{3465a4}
\definecolor{Plum}{HTML}{75507b}
\definecolor{Slate}{HTML}{555753}
\definecolor{ScarletRed}{HTML}{cc0000}
%Shadow
\definecolor{AluminiumShadow}{HTML}{babdb6}
\definecolor{ButterShadow}{HTML}{c4a000}
\definecolor{ChameleonShadow}{HTML}{4e9a06}
\definecolor{OrangeShadow}{HTML}{ce5c00}
\definecolor{ChocolateShadow}{HTML}{8f5902}
\definecolor{SkyBlueShadow}{HTML}{204a87}
\definecolor{PlumShadow}{HTML}{5c3566}
\definecolor{SlateShadow}{HTML}{2e3436}
\definecolor{ScarletRedShadow}{HTML}{a40000}

\newcounter{KortSet}
\newcounter{KortNummer}[KortSet]


%\usepackage{showframe}

\title{Dyrekort}
\begin{document}

\def\farver{{Aluminium},{Butter},{Chameleon},{Orange},{Chocolate},{SkyBlue},{Plum},{Slate},{ScarletRed}}

% Ark med farvede overskrifter til hvert sæt kort.
\thispagestyle{empty}
\setlength{\arrayrulewidth}{1mm}
\foreach \farve in \farver {% 
{\color{\farve} \HUGE
\hfill \Titel \hfill {\small printet: \today}
\hline
\vfill
}
}
\cleardoublepage

\foreach \farve in \farver {% Loop der laver flere sæt kort, med forskellig farve
\stepcounter{KortSet}

\tcbset{width=(\linewidth-7mm)/3,nobeforeafter,arc=4mm,
colframe=\farve,colback=white,
left=1mm,right=1mm,top=1mm,bottom=1mm,equal height group=parbox}
\DTLforeach*{kort}
{\tekst=Dyr, \billede=Billede}
{ % Kort
\stepcounter{KortNummer}
\begin{tcolorbox}
	\vspace{1mm}
	\Huge \centering \tekst  \\
	\vspace{2mm}
	\centering
 	\includegraphics[height=42mm]{\billede}
	\vfill \hfill
	{\small \arabic{KortNummer}}
\end{tcolorbox}
\hspace{1mm} % Vandret afstand mellem kort
\vspace{1mm} % Lodret afstand mellem kort
} % Kort slut

\cleardoublepage
} % Slut kort sæt loop
\end{document}
